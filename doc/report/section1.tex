\section{Programming for the Android Platform}

\subsection{What is Android?}
''Android'' as you likely know, is the name of a popular mobile device operating system. It is maintained and published by Google,
but released as an open spec, available to any manufacturer who wishes to build a device running it.

As a result of this, there exist a
wide variety of Android devices, and contingent with the continuing support of Google, applications written for the system are 
extremely platform-independent; that is, code written for Android will run without issue on nearly any Android device, and when it
does not run, the reason will be extremely clear and uncomplicated to address.

\subsection{Why Android?}
The reasons why the ATG makes sense as a mobile phone application are somewhat outside the scope of this document, but the issue of
operating system choice does necessitate some discussion.\par
First, the Android OS is a far more approachable development environment. Google helpfully offers a free IDE (Integrated Development Environment) to aid ease of development for their system. An IDE is a program that provides an array of tools to write, compile, and debug code, 
attempting to support your project ''from cradle to grave.'' This IDE is a large part of what makes this project possible, as it provides
essentially all the tools that we, as students, are incapable of developing for ourselves. In addition to boasting a friendly support
administration, Android and all its apps are written natively in a very common language: Java. Say what you will about interpreted
languages, but they do make development snappy and quick.\par
Compare this with IOS, the main competitor one might consider when writing a phone application. The iPhone is a \emph{closed} development
environment. This means that Apple has made it very hard for anyone not using Apple hardware and some version of OSX to write for their system.
Technically, it is not impossible. Apple maintains that one can purchase a license for an old version of OSX for cheap, install it with a
fresh ''manufacturer key,'' and get to work on an IOS app; but it cannot be denied that the support from Apple for non-OSX IOS developers
is decisively worse than that of Google for Android. An additional consideration is the native language of IOS apps: they're written
a language called... Swift? Honestly, I had to go look it up. Never heard of it in my life. I don't have any backround with it, and neither
do I believe have my teammates. Swift doesn't seem to be horribly different from standard C-family languages, but I'm certain it is different
enough to be a stumbling block. Overall, I wouldn't oppose the development of ATG for Android, but to do so would be beyond my skills.

\subsection{The Structure of an Android App}
From this section onward I shall assume that the reader has some elementary programming knowledge. If you find the next sections confusing
or encounter too many unknown words, it is advisable to read and work some introductory lessons in the Java programming language.
The ''Resources'' section at the end of this manual shall contain pointers to helpful references.\cite{javaOracle}

Every Android application is composed first and foremost of \verb|Activity|s. An \verb|Activity| is a developer-defined Java class which extends the 
Android API class \verb|Activity|. In order to do this, one must override all of \verb|Activity|'s abstract methods in the child class, implementing them
to actually do the things one wants the app to do. Specifically, there are five methods where one's code will be run:

\begin{enumerate}
\item{\verb|onCreate()|}
\item{\verb|onStart()|}
\item{\verb|onResume()|}
\item{\verb|onPause()|}
\item{\verb|onDestroy()|}
\end{enumerate}

Google's Android API documentation explains these far better than I might, so I shall simply direct you there.\cite{androidRefIndex} The important thing for our
purposes is that programming an application of any type is somewhat different than the programming one might learn in introductory classes.

\subsection{Asynchronous Programming}
On a phone, website, or any system more complicated than an introductory programming tutorial, code does not run in a single linear order.
The Android system will call the app's functions when it has decided the time is right. As such, when writing code, one must be constantly
aware of \emph{when} and \emph{where} the code will actually execute. Everything is implemented in terms of \emph{callbacks}, functions which
are called to adress specific circumstances. So, for example, when Android wants to load the app, it will look into our app code and find
the function \verb|onCreate()|, and will call that function, expecting that we have implemented that method to create our app. Thus, inside
the \verb|onCreate()| method, we should only do things that create the app in preparation for running, and nothing else. Otherwise, 
the code we write will be executed in unpredictable circumstances and will be very hard to debug and maintain. Asynchronous programming is
a fundamental concept in application programming. 

\subsection{Android Services, Permissions, and the Manifest}
Most services and utilities on Android are initialized and accessed in a fairly standard Java way, so in that aspect they are easy. However, many things an application might
wish to do are restricted by Android permissions controls; to surpass this, the programmer must follow some fairly specific guidelines on the use of permissions-protected 
functionality. Failure to do this will generally cause the app to crash and burn in a pile of unhandled \verb|SecurityException|s. More information on this topic may be found
in the Android Devloper Resources.\cite{androidPermissions}

\subsubsection{Manifest Permissions}
First, if the app needs to use functionality guarded by permission controls, it must declare so in it's \verb|AndroidManifest| file. This is a .xml file declaring a bunch of
things about the app, but the important point here is that outside the \verb|application| element of the file must be a line declaring that the app \verb|uses-permission|
followed by some valid permission. For example, if the app needs access to fine location data (like the ATG) then one of the permission lines will appear as:
\begin{verbatim}
<uses-permission android:name="android.permission.ACCESS_FINE_LOCATION" />
\end{verbatim}
On older versions of Android, this is all which is needed to acquire permissions. Unfortunately for deveopers, though, Google has taken a more progressivist stance on
permissions in newer editions. Originally, the manifest would declare a permission, and the user would be forced to grant all required permissions at install-time. However,
Android switched to a dynamic permissions system, where users can see what permissions an app requires at install-time, but don't necessarily grant them until runtime. This
means that all modern Android app code must check first whether the version of Android is new enough to use dynamic permissions, then if the user has already granted the
appropriate permission. If this check fails, the app must request the appropriate permissions and then implement a callback to receive the results and update the app state
appropriately. 

\subsubsection{Checking Permissions}
An app may check its own permissions with the \verb|checkSelfPermissions| method. This takes a \verb|Context| and \verb|Permission| value as arguments, and returns
a value from \verb|PackageManager| as to whether the system has granted the permission. So, code to check whether the system has fine location permissions would read:
\begin{minted}[breaklines]{java}
if(ContextCompat.checkSelfPermission(this, 
	Manifest.permission.ACCESS_FINE_LOCATION) == 
	PackageManager.PERMISSION_GRANTED) {
                   //do permissions-required stuff or set flags here 
        }
\end{minted}
In the ATG app, flags are set to track whether permissions have been gained. So the inside of the above code block would be as simple as setting the flag to true and calling
the initialization method.

\subsubsection{Requesting Permissions}
However, not all is fine and dandy. At least once in an app's lifetime, it won't have the requested permissions. In that case, it must request the permissions. However, the 
request will not be handled synchronously, as it requires a user prompt. Thus, an app must override \verb|onPermissionsRequestResult| in order to receive the result.
This is something of an involved process, which is better explained by the Android documentation, but I'll at least give the citation to the
developer resources on the subject.\cite{androidPermissions}


\documentclass{report}

\title{The A.T.G. on Android}
\author{Joseph A. Gerardot}
\date{\today}

\usepackage{palatino}
\usepackage[backend=biber]{biblatex}
\usepackage{hyperref}
\hypersetup{colorlinks}
\usepackage{bookmark}

\usepackage{xcolor}
\hypersetup{
    colorlinks,
    linkcolor={red!50!black},
    citecolor={blue!50!black},
    urlcolor={blue!80!black}
}

\addbibresource{ATGAndroidReport.bib}

\begin{document} 

\maketitle

\tableofcontents


\section*
\hrulefill
This document is intended to be a full manual, guide, and development record for the ATG Android application.
The ATG, as should be well known, stands for Automated Tour Guide, a device purposed to guide visually impaired 
students, faculty, parents, and other visitors around the campus without the use of an expensive mobility expert. \par
It has been long proposed that a phone app might be a suitable deployment of this technology, and now that it is coming to fruition
it is paramount that extremely good documentation be maintained along with it. It is fact that the DRC-ATG team's previous efforts have been
\emph{plagued}, not with bad programming, but lack of effective documentation. If we can successfully maintain and transfer
a knowledge base between teams, then this project may yet be completed.\par
With that in mind, this report shall not be static, but instead a living document. I shall write everything I know in it, but when 
later teams discover more, I expect them to integrate their knowledge with this report and add their own names to the title page.
Despite naysaying, I believe, with concerted effort, that this team truly can deliver a package of quality, usable software. Let us begin.

\section{Programming for the Android Platform}

\subsection{What is Android?}
''Android'' as you likely know, is the name of a popular mobile device operating system. It is maintained and published by Google,
but released as an open spec, available to any manufacturer who wishes to build a device running it.

As a result of this, there exist a
wide variety of Android devices, and contingent with the continuing support of Google, applications written for the system are 
extremely platform-independent; that is, code written for Android will run without issue on nearly any Android device, and when it
does not run, the reason will be extremely clear and uncomplicated to address.

\subsection{Why Android?}
The reasons why the ATG makes sense as a mobile phone application are somewhat outside the scope of this document, but the issue of
operating system choice does necessitate some discussion.\par
First, the Android OS is a far more approachable development environment. Google helpfully offers a free IDE (Integrated Development Environment) to aid ease of development for their system. An IDE is a program that provides an array of tools to write, compile, and debug code, 
attempting to support your project ''from cradle to grave.'' This IDE is a large part of what makes this project possible, as it provides
essentially all the tools that we, as students, are incapable of developing for ourselves. In addition to boasting a friendly support
administration, Android and all its apps are written natively in a very common language: Java. Now, say what you will about interpreted
languages, but they do make development snappy and quick.\par
Compare this with IOS, the main competitor one might consider when writing a phone application. The iPhone is a \emph{closed} development
environment. This means that Apple has made it very hard for anyone not using Apple hardware and some version of OSX to write for their system.
Technically, it is not impossible. Apple maintains that one can purchase a license for an old version of OSX for cheap, install it with a
fresh ''manufacturer key,'' and get to work on an IOS app; but it cannot be denied that the support from Apple for non-OSX IOS developers
is decisively worse than that of Google for Android. An additional consideration is the native language of IOS apps: they're written
a language called... Swift? Honestly, I had to go look it up. Never heard of it in my life. I don't have any backround with it, and neither
do I believe have my teammates. Swift doesn't seem to be horribly different from standard C-family languages, but I'm certain it is different
enough to be a stumbling block. Overall, I wouldn't oppose the development of ATG for Android, but to do so would be beyond my skills.

\subsection{The Structure of an Android App}
From this section onward I shall assume that the reader has some elementary programming knowledge. If you find the next sections confusing
or encounter too many unknown words, it is advisable to read and work some introductory lessons in the Java programming language.
The ''Resources'' section at the end of this manual shall contain pointers to helpful references.\cite{javaOracle}

Every Android application is composed first and foremost of \verb|Activity|s. An \verb|Activity| is a developer-defined Java class which extends the 
Android API class \verb|Activity|. In order to do this, one must override all of \verb|Activity|'s abstract methods in the child class, implementing them
to actually do the things one wants the app to do. Specifically, there are five methods where one's code will be run:

\begin{enumerate}
\item{\verb|onCreate()|}
\item{\verb|onStart()|}
\item{\verb|onResume()|}
\item{\verb|onPause()|}
\item{\verb|onDestroy()|}
\end{enumerate}

Google's Android API documentation explains these far better than I might, so I shall simply direct you there. The important thing for our
purposes is that programming an application of any type is somewhat different than the programming one might learn in introductory classes.

\subsection{Asynchronous Programming}
On a phone, website, or any system more complicated than an introductory programming tutorial, code does not run in a single linear order.
The Android system will call the app's functions when it has decided the time is right. As such, when writing code, one must be constantly
aware of \emph{when} and \emph{where} the code will actually execute. Everything is implemented in terms of \emph{callbacks}, functions which
are called to adress specific circumstances. So, for example, when Android wants to load the app, it will look into our app code and find
the function \verb|onCreate()|, and will call that function, expecting that we have implemented that method to create our app. Thus, inside
the \verb|onCreate()| method, we should only do things that create the app in preparation for running, and nothing else. Otherwise, 
the code we write will be executed in unpredictable circumstances and will be very hard to debug and maintain. Asynchronous programming is
a fundamental concept in application programming. 


\section{Application Structure}

Having now a cursory knowledge about how an Android app works, we can discuss what the ATG actually is, programmatically.
\subsection{ATG: The Basic Structure}
The ATG was first envisioned in a way very different than it is now. Initially, it was thought that an automated tour guide for the visually
impaired would be a very talkative device, speaking to the user at a nearly constant rate and giving extremely minute instructions. However,
it was found upon interviews with potential users that such a device would be considered annoying and patronizing. Consequently, the ATG was
re-imagined as a more hands-off sort of affair, occasionally telling the user their location, the names of buildings, or the locations of
bus stops, but leaving the actual walking and low-level navigation to the user himself, on the assumption that a visually impaired person has
dealt with their condition for years and can manage the evasion of minor obstacles themself.

Thus, both the ATG app and the legacy ATG work on a simple principle. They find the user's current location using GPS services and compare it
with a list of preset locations and radii. If the user is within a radius, the machine speaks some prescribed instructions informing
the user where they are and how they can navigate forward. These points (later called nodes) are organized into \emph{routes}, which in the 
legacy ATG are an ordered set of points but in the app are simply an unordered collection. 

When the user loads of the ATG, they have the opportunity to select a route on which to navigate, and the ATG loads the GPS coordinates, 
radii, and descriptions of the points on the route, then listens to the GPS position of the device in order to detect when to play the audio.

Thus, the ATG has a simple operation loop:

\begin{enumerate}
\item{Receive GPS position}
\item{Check if the GPS position is within the radius of any node on the route}
\item{If the position is in the radius, play the corresponding instructions}
\end{enumerate}
\subsection{Activities and Services}

The abstract structure established above, however, must be translated into an usable application on the Android platform. This necessarily 
breaks the app down into a set of \verb|Activity|s and \verb|Service|s. Recall that an \verb|Activity| is a chunk of code which can be loaded
and display to the screen, and a \verb|Service| is a chunk of code that can be launched and will do processing without displaying to the 
screen.

Every Android application has a \verb|MainActivity|, which is automatically loaded by Android when the user requests the app be opened. 
In the case of our application, this the menu selection screen. The menu selection screen then passes an \verb|Intent| when it comes time to launch the next screen.
This \verb|Intent| contains information for the Android operating system to know what thing to launch, and it can also have extra information passed along with it.
To start an \verb|Activity| named \verb|Test.class|, for example, one would use the following:
\begin{verbatim}
startActivity(new Intent(this, Text.class);
\end{verbatim}
All the screens of the application are accessed in this way. For a more complete

\subsection{User Interface}
Beyond simply opening screens, however, the app needs to accept user input for all the menus. This is accomplished by the use of the \verb|GestureDetector| class.
All the screens which accept user input simply initialize a \verb|GestureDetector| and overload their \verb|onMotionEvent| to pass the \verb|MotionEvent| to it.

In order to use \verb|GestureDetector|, one needs to implement the entire interface \verb|onGestureListener|. However, this app communicates with the user entirely through
left/right swipes and long presses of the screen. This means most of other types of input are unnecessary, so they have methods which do nothing. Implementing the methods, 
however, is required to stop the compiler from yelling at us. 

To maintain track of which item is currently selected, the screens with menus simply intialize a member array of the options and an index integer, and increment/decrement the
index on the appropriate user input. For convience, they all have a display update method to communicate the change to the user.

For more information on user interface in the Android Environment, see the references.\cite{gestureDetector}

\subsection{Location Services}
The core of the application is, of course, Google Location Services.


\section{Routes}
\subsection{What is a Route?}
The entire ATG revolves around routes, so it's worth spending some time to define what they are. In essence, a route is simply a list of
points, and radii around them. The legacy ATG uses a linked-list of points, where the points on the route are ordered. Programming it this way
had the unfortunate side-effect that the ATG would only be watching for the next point on the route, so if a user accidentally skipped one,
or the GPS didn't pick it up, the ATG would be unable to respond to any more points until the user walked back and hit the first one.
In programming the new ATG phone application, we've solved this by removing the idea that the points even have an order. The app now simply
checks every point in the route on every location update.

Having discussed the routes, it's simple to define what's in each point (or \emph{node}, as they will now be called.). The node needs to be
centered at a location, so it has a latitude and longitude, and it needs a radius to define how large of a circle it represents.

\subsection{Node Formatting}
The newest format for a node saves this data inside a .txt file, on the first line, separated by spaces. For example, to represent a node at
latitude 60, longitude 30, and with radius of seven meters, node.txt would look like:
\begin{verbatim}
60 30 7
\end{verbatim}
Importantly, the node.txt file can contain more data without causing issues. The latitude, longitude, and radius simply need to be the first
three numbers to appear in the file. In addition, there is an odd issue with the way Java handles floating-points, where specifying too much
precision will in any of the numbers will cause a crash. In order to prevent this, one should never write more than 12 decimal places
$(10^{-12})$ in the numbers for this file.

The other thing needed for every point in a route is a description: the instructions to give when the user is at that place. In the current
format, this is saved as plaintext in a file called ``speech.txt.'' Writing the description is a simple as opening up that file and 
transcribing what the device should say.

\subsection{Route Formatting}
Both of the above files compose a \emph{node}, the logical atom in a route. The current route format puts each node in its own subdirectory
on the Android filesystem. A route, then, is simply a subdirectory containing subdirectories, with a single file ``desc.txt,'' to hold the
description of the route itself for the selection screen. All the routes on a phone are contained in one top-level directory in the
Android External Storage drive, called ATG. Thus, the filesystem for a phone with ATG properly installed would look like so:\vspace{.1pt}
\dirtree{%
.0 ATG.
.1 route1.
.1 route2.
	.2 desc.txt.
	.2 node1.
	.2 node2.
	.2 node3.
	.2 node4.
	.2 node5.
		.3 node.txt.
		.3 speech.txt.
.1 route3.
.1 route4.
.1 static.
	.2 staticroute1.
	.2 staticroute2.
		.3 desc.txt.
		.3 node1.
		.3 node2.
}



\printbibliography[title=References]
\end{document}

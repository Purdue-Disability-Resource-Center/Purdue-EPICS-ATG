\documentclass{report}

\title{The A.T.G. as an Android App}
\author{Joseph A. Gerardot}
\date{\today}

\usepackage{palatino}
\usepackage[backend=biber]{biblatex}
\usepackage{hyperref}
\usepackage{bookmark}

\addbibresource{ATGAndroidReport.bib}

\begin{document} 

\maketitle

\tableofcontents


\section*
\hrulefill
This document is intended to be a full manual, guide, and development record for the ATG Android application.
The ATG, as should be well known, stands for Automated Tour Guide, a device purposed to guide visually impaired 
students, faculty, parents, and other visitors around the campus without the use of an expensive mobility expert. \par
It has been long proposed that a phone app might be a suitable deployment of this technology, and now that it is coming to fruition
it is paramount that extremely good documentation be maintained along with it. It is fact that the DRC-ATG team's previous efforts have been
\emph{plagued}, not with bad programming, but lack of effective documentation. If we can successfully maintain and transfer
a knowledge base between teams, then this project may yet be completed.\par
With that in mind, this report shall not be static, but instead a living document. I shall write everything I know in it, but when 
later teams discover more, I expect them to integrate their knowledge with this report and add their own names to the title page.
Despite naysaying, I believe, with concerted effort, that this team truly can deliver a package of quality, usable software. Let us begin.

\section{Programming for the Android Platform}

\subsection{What is Android?}
''Android'' as you likely know, is the name of a popular mobile device operating system. It is maintained and published by Google,
but released as an open spec, available to any manufacturer who wishes to build a device running it.

As a result of this, there exist a
wide variety of Android devices, and contingent with the continuing support of Google, applications written for the system are 
extremely platform-independent; that is, code written for Android will run without issue on nearly any Android device, and when it
does not run, the reason will be extremely clear and uncomplicated to address.

\subsection{Why Android?}
The reasons why the ATG makes sense as a mobile phone application are somewhat outside the scope of this document, but the issue of
operating system choice does necessitate some discussion.\par
First, the Android OS is a far more approachable development environment. Google helpfully offers a free IDE (Integrated Development Environment) to aid ease of development for their system. An IDE is a program that provides an array of tools to write, compile, and debug code, 
attempting to support your project ''from cradle to grave.'' This IDE is a large part of what makes this project possible, as it provides
essentially all the tools that we, as students, are incapable of developing for ourselves. In addition to boasting a friendly support
administration, Android and all its apps are written natively in a very common language: Java. Now, say what you will about interpreted
languages, but they do make development snappy and quick.\par
Compare this with IOS, the main competitor one might consider when writing a phone application. The iPhone is a \emph{closed} development
environment. This means that Apple has made it very hard for anyone not using Apple hardware and some version of OSX to write for their system.
Technically, it is not impossible. Apple maintains that one can purchase a license for an old version of OSX for cheap, install it with a
fresh ''manufacturer key,'' and get to work on an IOS app; but it cannot be denied that the support from Apple for non-OSX IOS developers
is decisively worse than that of Google for Android. An additional consideration is the native language of IOS apps: they're written
a language called... Swift? Honestly, I had to go look it up. Never heard of it in my life. I don't have any backround with it, and neither
do I believe have my teammates. Swift doesn't seem to be horribly different from standard C-family languages, but I'm certain it is different
enough to be a stumbling block. Overall, I wouldn't oppose the development of ATG for Android, but to do so would be beyond my skills.

\subsection{The Structure of an Android App}
From this section onward I shall assume that the reader has some elementary programming knowledge. If you find the next sections confusing
or encounter too many unknown words, it is advisable to read and work some introductory lessons in the Java programming language.
The ''Resources'' section at the end of this manual shall contain pointers to helpful references.\cite{javaOracle}

Every Android application is composed first and foremost of \verb|Activity|s. An \verb|Activity| is a developer-defined Java class which extends the 
Android API class \verb|Activity|. In order to do this, one must override all of \verb|Activity|'s abstract methods in the child class, implementing them
to actually do the things one wants the app to do. Specifically, there are five methods where one's code will be run:

\begin{enumerate}
\item{\verb|onCreate()|}
\item{\verb|onStart()|}
\item{\verb|onResume()|}
\item{\verb|onPause()|}
\item{\verb|onDestroy()|}
\end{enumerate}

Google's Android API documentation explains these far better than I might, so I shall simply direct you there. The important thing for our
purposes is that programming an application of any type is somewhat different than the programming one might learn in introductory classes.

\subsection{Asynchronous Programming}



\end{document}

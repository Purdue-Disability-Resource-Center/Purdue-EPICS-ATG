\section{User Manual}
Use of the ATG application is intended to be fairly intuitive, but as with everything, explict instructions are necessary. If in doubt, don't 
be afraid to simply restart the application, and remember to delete the settings.txt in the static folder if static routes are added or
removed.
\subsection{General Tips}
Some general guidelines for using the Android ATG application:
\begin{itemize}
\item{All menus in in the app use a system where left and right swipes navigate between options, and a long press anywhere on the screen
selects the current option.}
\item{Every menu option and operation should also output some audio, instructing the user as to what happened and how to proceed. If you do
something and don't hear anything, it didn't work.}
\item{The app is dependent on having the phone directory structure set up correctly. If experiencing issues, the first thing to check is
whether that directory has been changed in any way.}
\item{If crashes continue after a correction of the directory structure, the next thing is to get the Android crash logs. For this, one
needs either a third-party app or developer. Such a case would be a good time to bring the app back to the ATG team.}
\item{The app requests location data from Android every one second, but there's no guarantee it actually gets a result every second. In worse
conditions, the app might now respond to location data very quickly, or at all. If it receives data that Android believes is inaccurate, the
user will be warned and no further instructions will be given until connectivity is restored.}
\end{itemize}
\subsection{Installing Routes}
In order to load routes to a phone, one needs access to the phone's internal storage directory. The route should come in the form of folder,
and is likely sitting on the filesystem on a PC. In order to put the route on the phone, all that is necessary is to plug the Android phone 
into the computer, open the Android filesystem from the computer, and drag the route into the folder labeled ``ATG'' on the phone. If this
folder does not exist, create it.

\subsubsection{Static Routes}
It is also possible to install a route as \emph{static}. This means the route will not be in the normal route selection menu. Instead, it will
become an option, enableable in the settings menu, and if enabled it will be loaded whenever navigation is happening. In order to install a 
route as static, one must perform a similar process as the installation of a normal route, except instead of placing the route folder in
the ``ATG'' directory, the route folder must be place in the ``static'' directory, which is inside the ``ATG'' directory. If removing or adding
routes to the ``static'' directory, one should delete the file ``settings.txt.'' The ATG will regenerate this file when it is restarted, and
the user will need to revisit the settings to put the settings once more the way the user wants them. Refer to the section
on the ATG filesystem if this is confusing or unclear. 

\subsection{Application Structure}
In order to navigate backwards from any screen in the app, the user should simply press the Android phone's back button.
\subsubsection{Main Screen}
The main screen is the one loaded immediately when the app is launched. This reads the user the disclaimer, and provides a simple menu to
select between going to the settings screen, and to the route selection screen. As always, swipe right and left to browse the options and 
press the screen to select the current one.

\subsubsection{Settings Screen}
The settings screen is accessible from the main screen, and allows the user to toggle which static routes are active or inactive. This
set of options will be dynamically populated with the routes in the ``static'' folder, so there's no need to do anything further than drop
the routes into that folder. All options default to active, but user selections are persistent between app instances, so there's no need to
reset the options every time, unless the ``static'' directory has changed. If this is the case, the ``settings.txt'' should have been deleted,
and the settings will need to be reset within the app.

\subsubsection{Route Selection Screen}
The route selection screen is the screen allowing the user to choose the route on which to navigate. It reads off the name of the current
route, and reads off each new route as it is swiped to, again in a left/right system. The user presses the screen to select a route and begin
navigating on it. This screen also includes some debug information for ATG team and DRC users, but it shouldn't trouble the average user.

\subsubsection{Route Navigation Screen}
The route navigation screen is shown when the user is actually navigating on a route. The navigation tracking is actually run as a backround
service, so there should be no need to app open on the screen or even to keep the phone open at all, although keeping the phone open
\emph{might} improve performance. The screen, though, does output some debug information and displays a map with pins at all the route points,
both for novelty and debugging.

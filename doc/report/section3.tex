\section{Routes}
\subsection{What is a Route?}
The entire ATG revolves around routes, so it's worth spending some time to define what they are. In essence, a route is simply a list of
points, and radii around them. The legacy ATG uses a linked-list of points, where the points on the route are ordered. Programming it this way
had the unfortunate side-effect that the ATG would only be watching for the next point on the route, so if a user accidentally skipped one,
or the GPS didn't pick it up, the ATG would be unable to respond to any more points until the user walked back and hit the first one.
In programming the new ATG phone application, we've solved this by removing the idea that the points even have an order. The app now simply
checks every point in the route on every location update.

Having discussed the routes, it's simple to define what's in each point (or \emph{node}, as they will now be called.). The node needs to be
centered at a location, so it has a latitude and longitude, and it needs a radius to define how large of a circle it represents.

\subsection{Node Formatting}
The newest format for a node saves this data inside a .txt file, on the first line, separated by spaces. For example, to represent a node at
latitude 60, longitude 30, and with radius of seven meters, node.txt would look like:
\begin{verabatim}
60 30 7
\end{verbatim}
Importantly, the node.txt file can contain more data without causing issues. The latitude, longitude, and radius simply need to be the first
three numbers to appear in the file. In addition, there is an odd issue with the way Java handles floating-points, where specifying too much
precision will in any of the numbers will cause a crash. In order to prevent this, one should never write more than 12 decimal places
(10^-12) in the numbers for this file.

The other thing needed for every point in a route is a description: the instructions to give when the user is at that place. In the current
format, this is saved as plaintext in a file called ``speech.txt.'' Writing the description is a simple as opening up that file and 
transcribing what the device should say.

\subsection{Route Formatting}
Both of the above files compose a \emph{node}, the logical atom in a route. The current route format puts each node in its own subdirectory
on the Android filesystem. A route, then, is simply a subdirectory containing subdirectories, with a single file ``desc.txt,'' to hold the
description of the route itself for the selection screen. All the routes on a phone are contained in one top-level directory in the
Android External Storage drive, called ATG. Thus, the filesystem for a phone with ATG properly installed would look like so:'
\begin{verbatim}
ATG
|--route1
|--route2
	|--desc.txt
	|--node1
	|--node2
	|--node3
	|--node4
	|--node5
		|--node.txt
		|--speech.txt
|--route3
|--route4
|--static
	|--staticroute1
	|--staticroute2
\end{verbatim}


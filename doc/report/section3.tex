\section{Routes}
\subsection{What is a Route?}
The entire ATG revolves around routes, so it's worth spending some time to define what they are. In essence, a route is simply a list of
points, and radii around them. The legacy ATG uses a linked-list of points, where the points on the route are ordered. Programming it this way
had the unfortunate side-effect that the ATG would only be watching for the next point on the route, so if a user accidentally skipped one,
or the GPS didn't pick it up, the ATG would be unable to respond to any more points until the user walked back and hit the first one.
In programming the new ATG phone application, we've solved this by removing the idea that the points even have an order. The app now simply
checks every point in the route on every location update.

Having discussed the routes, it's simple to define what's in each point (or \emph{node}, as they will now be called.). The node needs to be
centered at a location, so it has a latitude and longitude, and it needs a radius to define how large of a circle it represents.

\subsection{Node Formatting}
The newest format for a node saves this data inside a .txt file, on the first line, separated by spaces. For example, to represent a node at
latitude 60, longitude 30, and with radius of seven meters, node.txt would look like:
\begin{verbatim}
60 30 7
\end{verbatim}
Importantly, the node.txt file can contain more data without causing issues. The latitude, longitude, and radius simply need to be the first
three numbers to appear in the file. In addition, there is an odd issue with the way Java handles floating-points, where specifying too much
precision will in any of the numbers will cause a crash. In order to prevent this, one should never write more than 12 decimal places
$(10^{-12})$ in the numbers for this file.

The other thing needed for every point in a route is a description: the instructions to give when the user is at that place. In the current
format, this is saved as plaintext in a file called ``speech.txt.'' Writing the description is a simple as opening up that file and 
transcribing what the device should say.

\subsection{Route Formatting}
Both of the above files compose a \emph{node}, the logical atom in a route. The current route format puts each node in its own subdirectory
on the Android filesystem. A route, then, is simply a subdirectory containing subdirectories, with a single file ``desc.txt,'' to hold the
description of the route itself for the selection screen. All the routes on a phone are contained in one top-level directory in the
Android External Storage drive, called ATG. Thus, the filesystem for a phone with ATG properly installed would look like \ref{ATGtree}.
\begin{Tree}
\caption{A properly configured ATG filesystem.}\label{ATGtree}
\dirtree{%
.0 ATG.
.1 route1.
.1 route2.
	.2 desc.txt.
	.2 node1.
	.2 node2.
	.2 node3.
	.2 node4.
	.2 node5.
		.3 node.txt.
		.3 speech.txt.
.1 route3.
.1 route4.
.1 static.
	.2 staticroute1.
	.2 staticroute2.
		.3 desc.txt.
		.3 node1.
		.3 node2.
}
\end{Tree}

\subsection{Route Creation}
It is hardly ever necessary, however, for one to create a route using the format alone. Several tools are available which aid the process of route creation. The best of them isthe Online Route Creator, available \href{https://epics.ecn.purdue.edu/drc/routeCreator.html}{here}. As a backup, there is a Python script available to convert routes from 
Excel files into ATG format. This isn't really the advised way to create routes, but it does exist if there are problems with the online route creator.

\subsubsection{Online Route Creation}
The Online Route Creator is, again, available \href{https://epics.ecn.purdue.edu/drc/routeCreator.html}{here}. In order to use it, one simply clicks on the displayed map to
create points, and clicks on the created pins or their entries in the node list in order to edit the description information. Then, one simply clicks the ``download'' button
in order to download the route. It will download as a .zip archive and must be unzipped before installing on the phone. 

In addition, it is possible to upload a route one has already downloaded for revision. To do this, one simply zips the route file and sends it up using the ``upload'' button on
the site. The route can then be edited and downloaded as normal.

\subsubsection{Python Route Creation}
As above, there is a Python script which can create routes as well, from nodes stored in Excel files. If one wishes to use this, one must first create a Excel file in the
proper format. The first cell (A1) of the first sheet will be interpreted as the description text. After that, every row has then latitude, longitude, radius, and description
text of the node, in that order. Obviously these are in different cells, to they occupy columns 1-4. The latitude and longitude should be in degrees, and the radius in meters.
Below is a working Excel route.\par
\begin{tabular}{|c|c|c|c|}
	\hline
	Route Description & & & \\ \hline
	1.0 & 1.0 & 10.0 & First node description \\ \hline
	2.0 & 3.0 & 10.0 & Second node description \\ \hline
	4.0 & 5.0 & 10.0 & Third node description \\ 
	\hline
\end{tabular}\par
When run through the Python script, this will create a route with three nodes, at the lat/long points specified, and with radius 10 and the descriptions typed in the relevant
cells.

To run the script, one must first install Python. This is incredibly tedious on Windows, so I won't bother with it here. Once installed, one simply invokes the Python
executable and submits the filename of the Excel file as an argument. Python will create a new folder to hold the route, which will be named route[FILENAME]. For example, if
the Excel file were named ``myroute.xlsx'' then the corresponding command to create the route would be
\begin{minted}{bash}
python buildRoute.py myroute.xlsx
\end{minted}
The output would be placed in a new folder called ``routemyroute.'' Note that this command must be run from the directory in which the Excel file is stored. This means the 
buildRoute.py file should also be in this directory.

\section{Program Structure}

Having now a cursory knowledge about how an Android app works, we can discuss what the ATG actually is, programmatically.
\subsection{ATG: The Basic Structure}
The ATG was first envisioned in a way very different than it is now. Initially, it was thought that an automated tour guide for the visually
impaired would be a very talkative device, speaking to the user at a nearly constant rate and giving extremely minute instructions. However,
it was found upon interviews with potential users that such a device would be considered annoying and patronizing. Consequently, the ATG was
re-imagined as a more hands-off sort of affair, occasionally telling the user their location, the names of buildings, or the locations of
bus stops, but leaving the actual walking and low-level navigation to the user himself, on the assumption that a visually impaired person has
dealt with their condition for years and can manage the evasion of minor obstacles themself.

Thus, both the ATG app and the legacy ATG work on a simple principle. They find the user's current location using GPS services and compare it
with a list of preset locations and radii. If the user is within a radius, the machine speaks some prescribed instructions informing
the user where they are and how they can navigate forward. These points (later called nodes) are organized into \emph{routes}, which in the 
legacy ATG are an ordered set of points but in the app are simply an unordered collection. 

When the user loads of the ATG, they have the opportunity to select a route on which to navigate, and the ATG loads the GPS coordinates, 
radii, and descriptions of the points on the route, then listens to the GPS position of the device in order to detect when to play the audio.

Thus, the ATG has a simple operation loop:

\begin{enumerate}
\item{Receive GPS position}
\item{Check if the GPS position is within the radius of any node on the route}
\item{If the position is in the radius, play the corresponding instructions}
\end{enumerate}
\subsection{Activities and Services}

The abstract structure established above, however, must be translated into an usable application on the Android platform. This necessarily 
breaks the app down into a set of \verb|Activity|s and \verb|Service|s. Recall that an \verb|Activity| is a chunk of code which can be loaded
and display to the screen, and a \verb|Service| is a chunk of code that can be launched and will do processing without displaying to the 
screen.

Every Android application has a \verb|MainActivity|, which is automatically loaded by Android when the user requests the app be opened. 
In the case of our application, this the menu selection screen. The menu selection screen then passes an \verb|Intent| when it comes time to launch the next screen.
This \verb|Intent| contains information for the Android operating system to know what thing to launch, and it can also have extra information passed along with it.
To start an \verb|Activity| named \verb|Test.class|, for example, one would use the following:
\begin{minted}{java}
startActivity(new Intent(this, Text.class);
\end{minted}
All the screens of the application are accessed in this way. For a more complete

\subsection{User Interface}
Beyond simply opening screens, however, the app needs to accept user input for all the menus. This is accomplished by the use of the \verb|GestureDetector| class.
All the screens which accept user input simply initialize a \verb|GestureDetector| and overload their \verb|onMotionEvent| to pass the \verb|MotionEvent| to it.

In order to use \verb|GestureDetector|, one needs to implement the entire interface \verb|onGestureListener|. However, this app communicates with the user entirely through
left/right swipes and long presses of the screen. This means most of other types of input are unnecessary, so they have methods which do nothing. Implementing the methods, 
however, is required to stop the compiler from yelling at us. 

To maintain track of which item is currently selected, the screens with menus simply intialize a member array of the options and an index integer, and increment/decrement the
index on the appropriate user input. For convience, they all have a display update method to communicate the change to the user.

For more information on user interface in the Android Environment, see the references.\cite{gestureDetector}

\subsection{Location Services}
The core of the application is, of course, Google Location Services. All the screens which deal with any location data work throughn this API, specifically by creating a
\verb|FusedLocationProvider| then then requesting location updates.

\subsubsection{The FusedLocationProviderClient}
The the \verb|FusedLocationProviderClient| is the key to all location services on Android. It is called ``Fused'' because it represents a fusion of location data from the GPS,
cell service, and wifi nodes. The \verb|FusedLocationProviderClient| handles this calculation entirely on its own, so all the programmer must do is call the methods to use it.
Unfortunately, the process is still somewhat complicated. The Android discussion on location-aware apps can help with any details not dicussed in this document.\cite{googleLocationServices}

\subsubsection{Initializing Location Services}
Before any location services can be used at all, the API must be initialized. It's fine to simply do this in the main screen and assume the later screens will follow, unless
the app includes any trickery with intent filters.

Issues related to permissions were covered in previous sections, but location services has a special extra layer of settings manipulation. In order to even use location
services at all, the app must have the correct permissions. However, in order for location data to actually come out correct, the user must set location settings in a way
that allows the app to operate. In the end, this means one must create a \verb|LocationSettingsRequest| and submit it to a checking method, and then pass a request for
resolution if it fails. This is again a rather tedious ordeal, so refer to the Android developer documentation for a better explanation.\cite{googleLocationServices}
Here is a working code block from app.
\begin{minted}[breaklines]{java}
locationRequest.setInterval(1000);       //setup for how fast we can receive locations. chosen arbitrarily
        locationRequest.setFastestInterval(1000);
        locationRequest.setPriority(LocationRequest.PRIORITY_HIGH_ACCURACY); //tell Google we want the best data

        LocationSettingsRequest.Builder builder = new LocationSettingsRequest.Builder() //get a package of all our location settings requests
                .addLocationRequest(locationRequest);
        SettingsClient client = LocationServices.getSettingsClient(this);               //get a client for the current settings on the phone
        Task<LocationSettingsResponse> task = client.checkLocationSettings(builder.build()); //create a new task checking if the two are compatible


        task.addOnFailureListener(this, new OnFailureListener() { //listen if task fails, and handle if it does
            @Override
            public void onFailure(@NonNull Exception e) {           //switch to deal with different errors
                int statusCode = ((ApiException) e).getStatusCode();
                switch (statusCode) {
                    case CommonStatusCodes.RESOLUTION_REQUIRED: { //in this case, need to ask user to change settings for us
                        try {
                            ResolvableApiException resolvable = (ResolvableApiException) e; //change type of exception
                            resolvable.startResolutionForResult(RouteSelect.this, 2); //ask Android to bring up the dialog to change. Check this line if there are problems.
                        }
                        catch (IntentSender.SendIntentException sendEx) {
                        RouteSelect.this.stats.updateGPSState(StatsManager.GPS_NEED_PERMISSION);
                        }
                        break;
                    }
                    case LocationSettingsStatusCodes.SETTINGS_CHANGE_UNAVAILABLE:
                        break;


                }
\end{minted}

\section{Application Structure}

Having now a cursory knowledge about how an Android app works, we can discuss what the ATG actually is, programmatically.
\subsection{ATG: The Basic Structure}
The ATG was first envisioned in a way very different than it is now. Initially, it was thought that an automated tour guide for the visually
impaired would be a very talkative device, speaking to the user at a nearly constant rate and giving extremely minute instructions. However,
it was found upon interviews with potential users that such a device would be considered annoying and patronizing. Consequently, the ATG was
re-imagined as a more hands-off sort of affair, occasionally telling the user their location, the names of buildings, or the locations of
bus stops, but leaving the actual walking and low-level navigation to the user himself, on the assumption that a visually impaired person has
dealt with their condition for years and can manage the evasion of minor obstacles themself.

Thus, both the ATG app and the legacy ATG work on a simple principle. They find the user's current location using GPS services and compare it
with a list of preset locations and radii. If the user is within a radius, the machine speaks some prescribed instructions informing
the user where they are and how they can navigate forward. These points (later called nodes) are organized into \emph{routes}, which in the 
legacy ATG are an ordered set of points but in the app are simply an unordered collection. 

When the user loads of the ATG, they have the opportunity to select a route on which to navigate, and the ATG loads the GPS coordinates, 
radii, and descriptions of the points on the route, then listens to the GPS position of the device in order to detect when to play the audio.

Thus, the ATG has a simple operation loop:

\begin{enumerate}
\item{Receive GPS position}
\item{Check if the GPS position is within the radius of any node on the route}
\item{If the position is in the radius, play the corresponding instructions}
\end{enumerate}
\subsection{Activities and Services}

The abstract structure established above, however, must be translated into an usable application on the Android platform. This necessarily 
breaks the app down into a set of \verb|Activity|s and \verb|Service|s. Recall that an \verb|Activity| is a chunk of code which can be loaded
and display to the screen, and a \verb|Service| is a chunk of code that can be launched and will do processing without displaying to the 
screen.

Every Android application has a \verb|MainActivity|, which is automatically loaded by Android at when the user requests the app be opened. 
In the case of our application, this the menu selection screen. 
